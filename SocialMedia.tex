%% Basierend auf einer TeXnicCenter-Vorlage von Tino Weinkauf.
%%%%%%%%%%%%%%%%%%%%%%%%%%%%%%%%%%%%%%%%%%%%%%%%%%%%%%%%%%%%%%

%%%%%%%%%%%%%%%%%%%%%%%%%%%%%%%%%%%%%%%%%%%%%%%%%%%%%%%%%%%%%
%% HEADER
%%%%%%%%%%%%%%%%%%%%%%%%%%%%%%%%%%%%%%%%%%%%%%%%%%%%%%%%%%%%%
\documentclass[a4paper,oneside,10pt]{report}
% Alternative Optionen:
%	Papiergr��e: a4paper / a5paper / b5paper / letterpaper / legalpaper / executivepaper
% Duplex: oneside / twoside
% Grundlegende Fontgr��en: 10pt / 11pt / 12pt



%% Deutsche Anpassungen %%%%%%%%%%%%%%%%%%%%%%%%%%%%%%%%%%%%%
\usepackage[ansinew]{inputenc} 
\usepackage[T1]{fontenc} 
\usepackage[ngerman]{babel}
\usepackage{csquotes} 


\usepackage{url}


\usepackage{microtype}
%\usepackage{lmodern} %Type1-Schriftart f�r nicht-englische Texte
%% Packages f�r Grafiken & Abbildungen %%%%%%%%%%%%%%%%%%%%%%
\usepackage{graphicx} %%Zum Laden von Grafiken
%\usepackage{subfig} %%Teilabbildungen in einer Abbildung
%\usepackage{tikz} %%Vektorgrafiken aus LaTeX heraus erstellen
\usepackage{lmodern}

\usepackage{hyperref}

\usepackage[ 
style=authoryear-ibid, autocite=footnote, 
]{biblatex} 
\bibliography{Literatur} 

\renewcommand*{\mkbibnamelast}[1]{\textsc{#1}} 
\defbibheading{bibliography}{% 
\section*{Literaturverzeichnis}} 

%% Zeilenabstand %%%%%%%%%%%%%%%%%%%%%%%%%%%%%%%%%%%%%%%%%%%%
%\usepackage{setspace}
%\singlespacing        %% 1-zeilig (Standard)
%\onehalfspacing       %% 1,5-zeilig
%\doublespacing        %% 2-zeilig


%% Andere Packages %%%%%%%%%%%%%%%%%%%%%%%%%%%%%%%%%%%%%%%%%%
%\usepackage{a4wide} %%Kleinere Seitenr�nder = mehr Text pro Zeile.
%\usepackage{fancyhdr} %%Fancy Kopf- und Fu�zeilen
%\usepackage{longtable} %%F�r Tabellen, die eine Seite �berschreiten


%%%%%%%%%%%%%%%%%%%%%%%%%%%%%%%%%%%%%%%%%%%%%%%%%%%%%%%%%%%%%
%% Optionen / Modifikationen
%%%%%%%%%%%%%%%%%%%%%%%%%%%%%%%%%%%%%%%%%%%%%%%%%%%%%%%%%%%%%

%\input{optionen} %Eine Datei 'optionen.tex' wird hierf�r ben�tigt.
%% ==> TeXnicCenter liefert m�gliche Optionendateien
%% ==> im Vorlagenarchiv mit (Datei | Neu von Vorlage...).
\setlength{\marginparwidth}{1.2in}
\let\oldmarginpar\marginpar
\renewcommand\marginpar[1]{\-\oldmarginpar[\raggedleft\footnotesize #1]%
{\raggedright\footnotesize #1}}


%%%%%%%%%%%%%%%%%%%%%%%%%%%%%%%%%%%%%%%%%%%%%%%%%%%%%%%%%%%%%
%% DOKUMENT
%%%%%%%%%%%%%%%%%%%%%%%%%%%%%%%%%%%%%%%%%%%%%%%%%%%%%%%%%%%%%
\begin{document}

\pagestyle{empty} %%Keine Kopf-/Fusszeilen auf den ersten Seiten.

%% Deckblatt %%%%%%%%%%%%%%%%%%%%%%%%%%%%%%%%%%%%%%%%%%%%%%%%
%% Die einfache Version:
\title{Twitter, Facebook \& Co. -- Wie soziale Medien die Kommunikation des 21. Jahrhunderts revolutionieren}
\author{Jonas Schneider}
%\date{} %%Wenn kommentiert, wird das aktuelle Datum verwendet.
\maketitle

%% Inhaltsverzeichnis %%%%%%%%%%%%%%%%%%%%%%%%%%%%%%%%%%%%%%%
\tableofcontents %Inhaltsverzeichnis
\cleardoublepage %Das erste Kapitel soll auf einer ungeraden Seite beginnen.

\pagestyle{headings} %%Ab hier die Kopf-/Fusszeilen



\chapter{Einleitung}




\chapter{Die Entstehung eines Internetgiganten}
ii.	Bsp: indymedia
b.	Soziale Netzwerke
i.	als Unterkategorie der Medien
ii.	Facebook/Sch�lerVZ/Lin/startkedIn/Xing
c.	Blogs

Das Entstehen von sozialen Medien bzw. Netzwerken wurde ma�geblich durch technologische Fortschritte gepr�gt. 




\section{Das WWW entsteht}
Der Grundstein zur Entstehung erfolgreicher sozialer Netzwerke muss daher weit vor ihrem erstmaligen Auftreten gelegt worden sein; ohne Zweifel stellt die Entwicklung des Internets selber den Anfang dieser Kette von Ereignissen dar. 
Nach anf�nglichen Versuchen diverser Forschungseinrichtungen, Mitte der 80-er Jahre ein Netzwerk zum Datenaustausch zwischen Forschungsstandorten, unter anderem denen des US-Milit�rs, gelang es Tim-Berners Lee und seinen Kollegen am Kernforschungsinstitut CERN in der Schweiz, 1989 die ersten ,,Webseiten'' zu erstellen, anzuzeigen und elektronisch weiterzuverbreiten.\footcite[Siehe dazu auch den Artikel][in dem die Entstehung des Webs aus der Sicht einer Forschungseinrichtung in den USA geschildert wird.]{slac_earlyweb}
Bereits in dieser Anfangszeit, und sogar schon davor, wurde der Wunsch nach Interaktion seitens der Benutzer laut, was damals jedoch aufgrund technischer Beschr�nkungen nicht zu realisieren war.
Mit der Verbreitung des World Wide Webs in der Welt zeichnete sich immer schneller ab, dass die dezentralisierte Struktur des Internets eines seiner gr��ten St�rken war; es gab und gibt keine Autorit�t, �ber die alle Verbindungen zu Webseiten ablaufen, die Kommunikation erfolgt direkt zwischen den beiden Kommunikationspartnern.
Diese dezentrale Struktur bezog sich allerdings nur auf die Gesamtheit des Webs; eine Webseite wurde dagegen nur von genau einem Computer �bertragen, an den alle Benutzer der Webseite ihre Anfragen richteten. Diese Beschr�nkung in Kombination mit dem damaligen Stand der Technik lie� es nicht zu, gr��ere Anwendungen, die �ber eine einzige Internetseite aufgerufen werden, zu entwickeln, da die damaligen Computer dem hohen Rechenaufwand nicht standhielten und es nicht m�glich war, diesen auf mehrere Computer zu verteilen.




\section{Erste soziale Netzwerke}
Ein Ansatz zur L�sung dieses Verteilungsproblems stellte die Grundlage f�r die ersten sozialen Netzwerke dar: anstatt alle Benutzer auf einer Webseite zu konzentrieren, verteilte man sie auf viele Rechner, die jeweils einen anderen Webseiten-,Namen' hatten. Dieser ,Name' hatte die Form 
\begin{flushleft} {\tt http://benutzername.portalname.com} \end{flushleft}
F�r die Computer des Webs war damit spezifiziert, dass der Benutzer zwar das Portal {\tt portalname.com} besuchen m�chte, jedoch nicht dessen Haupt-Webseite, sondern die Unter-Webseite {\tt benutzername}.
Die Verwendung dieser sogenannten Subdomains erm�glichte die Lastverteilung auf mehrere Rechner und damit die Bereitstellung eines gr��eren Gesamtnetzwerks.
Der gr��te Nutzer dieser Technik war \emph{Geocities}, ein Portal, das es seinen Benutzern erm�glichte, eigene Internetseiten zu erstellen und sie zu ver�ffentlichen.
Dabei waren, ganz im Zeichen der dezentralisierten Struktur des Webs, keine Beschr�nkungen bez�glich des Inhalts gesetzt; Benutzer hatten die M�glichkeit, unter Verwendung einer Webseiten-Beschreibungssprache (HTML), die auch noch heute f�r s�mtliche Webseiten verwendet wird, die eigenen Seiten von Grund auf mit eigenem Design und eigenen Inhalten zu f�llen.

Diese Freiheit sollte sowohl Triumph als auch Niedergang von Geocities und �hnlichen Netzwerken bedeuten. In ihrer Bl�tezeit sch�tzten die Benutzer die gro�en Freiheiten, schufen eigene Webseiten mit teilweise gelungenen, teilweise grauenhaften\marginpar{Beispiele} Designs, Formen und Farben. Diese Freiheiten zogen einen Wettbewerb mit sich, es herrschte offene Konkurrenz um die informativste, die bestaussehende und die aktuellste Seite. Dieser Wettbewerb, womit kein kommerzieller Wettbewerb impliziert werden soll, f�rderte die Entwicklung neuer Technologien ungemein und brachte der Geocities-Community\marginpar{Der Begriff \emph{Community} (engl. Gemeinde) bezeichnet die Nutzerschaft einer Internetplattform.} ein Gef�hl des Fortschritts und der Individualisierung.

Doch alsbald kristallierte sich die Schattenseite des Geocities-Modells heraus, die paradoxerweise aus dem gleichen Grundprinzip hervorgeht: die uneingeschr�nkte Freiheit der Benutzer, die Webseiten nach Lust und Laune zu gestalten und zu entwickeln, lie� es nicht zu, computerunterst�tzte Netzwerke zu entwickeln, die die einzelnen Seiten verbanden und ihnen einen maschinenunterst�tzten Zusammenhalt erm�glichten. Nachrichten, Fotos, aktuelle Meldungen und Beitr�ge von Freunden oder Bekannten verblieben auf deren Webseite, wenn sich die Betroffenen �berhaupt die M�he machten, ihre Webseite mit den teilweise komplizierten Bearbeitungsmethoden auf den Neuesten Stand brachten. Es war kein ,Strom' von Nachrichten, der die neuesten Beitr�ge und Informationen von und zu Freunden anzeigte und somit den Beginn einer Interaktion erm�glichte, verf�gbar. 

Auch lief die Kommunikation zwischen den Benutzern oft gar nicht �ber die Plattformen ab; abgesehen von Webseiten, die sowieso keinerlei Interaktion erm�glichen, und den meist schlecht besuchten Chatr�umen der Plattformen, war die dominante elektronische Kommunikationsform noch immer die E-Mail, eine Technologie, die bereits �lter als das Web selbst ist. Das Versenden von Nachrichten an andere Benutzer, heute ein Grundstein sozialer Netzwerke, lief deshalb komplett an den ersten sozialen Netzwerken vorbei; dies ist nur ein weiterer Faktor, der zur allm�hlichen ,,Zersetzung'' der Benutzerschaft dieser Plattformen f�hrte. Auch das Konzept der \emph{Webringe}, die Seiten zu einem gegebenen Thema verbinden, zusammenhalten und den Benutzern komfortabel verf�gbar machen sollten, fand keine gro�e Zustimmung, so dass die Inhalte im Internet Ende der 90er noch immer verstreut und ungegliedert waren.




\section{Das Scheitern der Spekulationsblase}
Der ausschlaggebende Faktor f�r den Fall der ersten Generation sozialer Netzwerke war jedoch das Platzen der Dotcom-Bubble, der Spekulationsblase im IT-Sektor. Ab 1995 beschleunigte sich das Wachstum des Marktsektors der IT- und Internetbranche sehr stark. Viele Anlieger investierten Geld in zukunftsorientierte, innovative IT-Unternehmen, aber auch in Unternehmen, die sie f�r solche hielten. Die hohe Investitionsbereitschaft der Anleger f�hrte zu absurden und irrationalen Investitionen; so beobachteten beispielsweise Firmen, die ihrem Namen ein ,.com' anh�ngten, einen rapiden Anstieg der eigenen B�rsenkurse.\footcite{prefix_investing} Doch die Unternehmen handelten jahrelang unwirtschaftlich und unprofitabel unter der Pr�misse, sich durch die jetzigen Ausgaben \emph{sp�ter} hohe Marktanteile und Profite sichern zu k�nnen. Zeitweise Einnahmen durch Kapitalgeber wurden f�r das Aufkaufen kleinerer Firmen verwendet, die Aktienkurse stiegen stark, obwohl sich die Firmen immer weiter verschuldeten.

Doch dann platze die Blase. Am 10. M�rz 2000 verzeichnete der NASDAQ-Index den h�chsten Kurs der Branche. In den folgenden 12 Monaten verlor er 50\%. �ber den genauen Ausl�ser wird noch immer spekuliert, die darauffolgende Kettenreaktion war jedoch definitiv fatal. Anleger zogen ihr Geld aus dem Markt, Kredite mussten gek�ndigt werden, und viele Firmen sahen sich mangels Gewinn auf dem Trockenen, mit der Insolvenz als einzigem Ausweg.

In der amerikanischen Gesellschaft wurde das Ansehen der Internetbranche stark besch�digt. Internet-Unternehmer wurden als Erbauer von Luftschl�ssern gesehen, die mit ihren teils guten Ideen jedoch nicht auf Erfolg stie�en, das Vertrauen in das Internet als Medium der Zukunft war ersch�ttert.

In den darauffolgenden Jahren handelten die meisten neu gegr�ndeten Technologiefirmen wieder unter traditionellen Ma�st�ben, die vorsahen, zun�chst einen Gewinn zu erzielen und mit diesem das Unternehmen auszubauen.








\section{Die Renaissance der sozialen Netzwerke}
Das allgemein gefasste Ziel sozialer Netzwerke besteht, wie bereits oben erw�hnt, im Aufbauen einer gro�en Anzahl von Benutzern; ein soziales Netzwerk macht f�r potenzielle Benutzer wenig Sinn, wenn kein Bekannter es bereits verwendet. Verwenden dagegen viele Bekannte das Netzwerk bereits, so ist der Sprung zur eigenen Anmeldung nicht mehr weit. Dieses virale\marginpar{Als \emph{viral} werden Angebote oder Webseiten bezeichnet, die sich durch einen Schneeballeffekt verbreiten.} Konzept war und ist die Grundidee sozialer Netzwerke; war einmal eine Art \emph{kritische Masse} an Benutzern �berschritten, wuchs das Netzwerk quasi von selbst. Mit einer sehr hohen Benutzerzahl wurde das Netzwerk dann auch f�r Werbeinteressenten attraktiv; eine hohe Durchdringungsrate in der Bev�lkerung half dabei.


Nun war leider die Idee, die gesamte Bev�lkerung mit einem sozialen Netzwerk durchdringen zu wollen, kl�glich mit der Dotcom-Bubble zu Grabe getragen worden. Doch man hatte bereits einen neuen Plan: Anstatt das Produkt f�r die gesamte �ffentlichkeit anzubieten, wie es die meiten Unternehmen davor taten, w�hlte man gewisserma�en die kleinere Herausforderung. Man spezialisierte sich auf einzelne gesellschaftliche Gruppen und stellte exklusiv f�r deren Mitglieder Gruppen eine Kommunikationsplattform bereit. Dadurch konnte die bereits angesprochene kritische Masse sehr viel leichter erreicht werden, da die Gesamtheit der zu erreichenden Nutzer wesentlich kleiner war. 


Dieses Konzept verwendete beispielsweise \emph{Match.com} als Dating- aber auch Kommunikationsplattform f�r Singles, sowie das deutsche Netzwerk \emph{Sch�lerVZ}, das nur Sch�ler anspricht. Die Unternehmen konnten Werbetreibenden eine demografische Spezialisierung anbieten, die ihnen erlaubte, effektiv und erfolgreich Werbung zu schalten. Der Plan der Unternehmer ging auf. 


Einer dieser Unternehmer war Mark Zuckerberg. Er verfolgte den angesprochenen Plan: ein soziales Netzwerk zur Kommunikation, nur f�r Harvard-Studenten. Es hatte auch einen Namen aus der Uni-Welt\marginpar{Ein \emph{facebook} im w�rtlichen Sinne bezeichnet ein Verzeichnis mit Fotos und Namen von Studenten.}: \emph{The Facebook}, der Name wurde sp�ter zu \emph{Facebook} verk�rzt.  Studenten stellen dabei objektiv gesehen die perfekte demografische Gruppe f�r soziale Netzwerke dar: noch jung genug, um sehr viel Wert auf soziale Kontakte, Trends und dergleichen zu legen, aber dennoch alt genug, um zum Einen kaufkr�ftig zu sein, zum Anderen aber auch etwas zu Sagen zu haben, um die Kommunikation nicht auf der Stelle treten zu lassen. Innerhalb kurzer Zeit wurde Facebook zum Trend und zum Muss f�r alle Studenten auf dem Campus.





\section{Das Prinzip Facebook}
Hier spielte Mark Zuckerberg und seinen Kollegen ein Beieffekt des Unilebens in die H�nde: alle Harvard-Studenten, und nur sie, besa�en E-Mail-Addressen der Form {\tt name@harvard.edu}. Anhand dieser Addresse konnte so mit einfachsten technischen Mitteln der Zugriff nur f�r Harvard-Studenten freigegeben werden. Diese Exklusivit�t, die Benutzer au�erhalb von Harvard ausschloss, f�rderte den Trendstatus von Facebook \emph{innerhalb} Harvards noch weiter.


Doch Facebook ging auch den n�chsten Schritt, den Schritt, den keiner vor ihnen gewagt hatte: sie expandierten. Auch hier war das Unikonzept ein Erfolgsrezept. Denn Harvard war nicht die einzige Uni, die spezielle E-Mail-Addressen an ihre Studenten verteilte. Hier zeigt sich auch ein weiterer Vorteil der demografischen Gruppe der Studenten; sie sind zum Einen gro� und allumfassend, da �berregional, zum Anderen aber stark regional fokussiert (auf die jeweilige Uni). Diese Dualit�t erm�glichte es Facebook, seine Plattform gezielt f�r einzelne Colleges freizuschalten, die dann durch Beziehungen zu Studenten in bereits freigeschalteten Einrichtungen eine starke Initialz�ndung erhielten. Bei der Expansion musste nur bedacht werden, die Geschwindigkeit so weit zu drosseln, dass die kritische Masse an Benutzern nicht unterschritten wurde.


Schlie�lich wurde Facebook USA-weit f�r Studenten freigegeben. Dann folgte der risikoreichste  Sprung in der Firmengeschichte; die �ffnung f�r die gesamte Gesellschaft. Doch durch den Trendstatus innerhalb der Studentengemeinschaft und die daraus folgende gro�e Marktdurchdringung gelang es Facebook, sich zum f�hrenden sozialen Netzwerk aufzuschwingen. 

Heute z�hlt es hinter Google zur zweitmeistbesuchten Webseite der Welt.

Doch die Nutzer von Facebook sind nicht nur in ihrer Zahl au�erordentlich. Fast drei Viertel der Benutzer besuchen die Seite t�glich. Diese Zahl ist f�r eine durchschnittliche Webseite utopisch. Selbst die Beliebtesten\marginpar{Welche?} konnten bis dato nur mit Besuchsfrequenzen von mehreren Tagen bis hin zu einer Woche aufwarten. 

Dieses Nutzerverhalten, die extrem hohe Besuchsfrequenz und die gro�e Rolle, die das Netzwerk im Leben der Nutzer spielt, sind die Symptome eines grunds�tzlichen Wandels des Kommunikationsverhaltens der Generation Facebook.



%%%%%%%%%%%%%%%%%%%%%%%%%%%%%%%%%%%%%%%%%%%%%%%%%%%%%%%%%%%%%%%%%%%%%




\chapter{Kommunikation im Wandel}
Die Antwort ist offensichtlich, aber doch nicht trivial. Der Sinn von Facebook ist es nicht, mit m�glichst vielen Menschen oder Bekannten verbunden zu sein. Es geht um Kommunikation, man will sehen, was die Freunde gerade tun, was sie m�gen, mit wem sie Pinnwandeintr�ge\marginpar{Das sollte erkl�rt werden} austauschen. Diese Form der Interaktion, die durchaus auch als weniger intensive bzw. passive Form des Stalkings bezeichnet werden kann, pr�gt den Umgang vieler Facebook-Nutzer untereinander. 
Der Reiz der Informationen, die andere �ber sich preisgeben, ist deshalb weniger in den Informationen selber, die oft nur sekund�r wichtig oder gar banal\marginpar{Beispiele} sind, zu finden, sondern vielmehr in der Tatsache, dass diese Informationen von Menschen stammen, die der Benutzer kennt. Dieses Faktum, das offensichtlich erscheinen mag, ist der Grundstein f�r die sehr hohe Besuchsfrequenz und die virale Verbreitung\marginpar{Erkl�rung} von Facebook unter Jugendlichen der heutigen Gesellschaft.
\section{Verbreitung}
\subsection{Nerds -> Jugendliche}
Einfach wegen viral und so?

\subsection{Jugendliche -> Wirtschaft}
weil Zielgruppe

\subsection{Wirtschaft -> Politik}
So ist es mittlerweile nat�rlich noch nicht Gang und Gebe, aber doch in Einzelf�llen m�glich, dass �ber Twitter Regierungsinformationen exklusiv verbreitet werden. 
Der Regierungssprecher Steffen Seibert beispielsweise twittert als @RegSprecher (http://twitter.com/regsprecher).




\section{Casual Communication}
Dieses Wort gibt es nicht, aber seine �bersetzung zeigt deutlich, was damit gemeint ist. \emph{Casual Communication}, englisch f�r \enquote{beil�ufige Kommunikation}, beschreibt einen grunds�tzlichen Wandel der Kommunikation.

Dieser Wandel ist allerdings keine Kehrtwende. \marginpar{Stimmt das Gelaber �berhaupt?} Schon in der Antike, wo milit�rische Pl�ne mangels Kommunikationsm�glichkeiten nur aufw�ndig und langwierig abgewickelt werden konnten (wenn �berhaupt), sehnte man sich nach k�rzeren Nachrichtenlaufzeiten.
Postkutschen stellten eine sp�tere Stufe der Entwicklung dar, die Einl�utung des modernen Kommunikationszeitalters begann mit der Erfindung des Telegrafen.

Vom fr�heren aufw�ndigen Briefe schreiben, �ber Telegramme und das Telefon, �ber E-Mail und Facebook-Nachrichten, sind wir nun bei 160-Zeichen-SMS angelangt. Doch geht es noch k�rzer? Nat�rlich!

Das h�tte das Motto des Kurznachrichtendienstes \emph{Twitter} sein k�nnen. Das Prinzip ist einfach --- jeder Nutzer kann Nachrichten an all diejenigen verschicken, die ihm auf Twitter \emph{folgen} (seine Follower). Die Grundfrage, die diese Nachrichten beantworten sollen, ist \enquote{What�s happening?}. Der Clou: Nachrichten d�rfen maximal 140 Zeichen lang sein. Diese extrem geringe Menge an Text verlangt es von den Benutzern ab, in ihren Nachrichten entweder auf den Punkt zu kommen, oder den Punkt komplett wegzulassen.
Dieser letzte Fall stellt f�r Viele die negative und �berwiegende Sicht auf Twitter dar --- oft enthalten die kurzen Nachrichten nur geringe Menge an Information.

Doch trotzdem stellt sich die Frage: \blockquote[{\cite[S. 20]{Twitterbuch_SN}}]{Was macht jedoch das Besondere und damit den Erfolg einer Seite aus, die doch angeblich nur mit ,was machst Du gerade'-Nachrichten bef�llt wird?}



Doch die Nutzer machen weit mehr draus...




\section{Aufstieg der Blogs}
Doch eine Form der sozialen Medien �berlebte den Dotcom-Bust, und gewann durch ihn sogar noch an Aufschwung: Weblogs oder Blogs.
Weblogs stellen die logische Konsequenz aus den W�nschen nach freiheitlicher Gestaltung, Interaktivit�t und Aktualit�t der damaligen Internetnutzer dar.
Sie sind eine Mischform zwischen Webseite und Kolumne, eine Webseite, die sich wie eine Kolumne st�ndig mit neuen Inhalten pr�sentiert.
Dabei wird nicht die komplette Seite durch den Eigent�mer st�ndig umgebaut, sondern er pflegt nur neue Inhalte ein, die die Leser des Blogs sehen k�nnen.
Meist handelt es sich dabei um Beitr�ge zu einem bestimmten Thema, oder das Blog wird als virtuelles Tagebuch des Eigent�mers gef�hrt.
Ab 1999 stieg die Zahl der Weblogs aktiv an, heute gibt es zu jedem erdenklichen Themenbereich mindestens eine Handvoll Weblogs.





\section{Institutionalisierung der sozialen Netzwerke}
Die Erfassung der sonst eher tr�gen Politik ist f�r ein soziales Netzwerk schon die h�chste Errungenschaft im Normalzustand. Bei einigen wenigen weit verbreiteten Netzwerken wie Facebook und Twitter aber ist die �ffentliche Einstellung gegen�ber der Plattform eine weitaus weniger distanzierte, als sie sonst zu neuen Medien ist.
Mittlerweile weit verbreitete Plattformen wie Twitter und Facebook werden zusehends als Medien im traditionellen Sinne angesprochen. So werden als Quelle f�r Nachrichten immer h�ufiger Tweets\marginpar{\emph{Tweets} sind von Twitter-Nutzern verfasste Kurzbeitr�ge.} oder Facebook-Kommentare herangezogen. Die Plattformen selber verlieren dabei ihren Produktcharacter -- w�hrend normalerweise von \enquote{kommerziellen Plattformen} gesprochen wird, wird Facebook wie ein Fernsehsender oder ein Magazin dargestellt.
\begin{itemize}
  
\item \enquote{Weil sie auf der Arbeit zuviel Zeit bei Facebook und mit Internet-Surfen verbrachte, hat eine Angestellte einer griechischen Fluggesellschaft ihren Job verloren.} --- \emph{sueddeutsche.de}
      \footnote{\url{http://newsticker.sueddeutsche.de/list/id/1143146}}


\item \enquote{Auf Facebook hatte P�hse seine Vereinsmitgliedschaft selbst �ffentlich gemacht.} --- \emph{Zeit Online}
      \footnote{\url{http://blog.zeit.de/stoerungsmelder/2011/04/22/werder-bremens-unerwunschter-npd-fan_6132}}


\item \enquote{Facebook l�sst nicht locker: Immer wieder liegen Beitritts-Einladungen oder -Erinnerungen im Postfach, obwohl man dem Netzwerk nie seine E-Mail-Adresse gegeben hat.} --- \emph{sueddeutsche.de}
      \footnote{\url{http://newsticker.sueddeutsche.de/list/id/1144044}}

\end{itemize}
Das Besuchen von Facebook ist hier kein Internet-Surfen, das Facebook-Profil eines Politikers gilt als Beweis f�r die offzielle Verk�ndung einer Vereinsmitgliedschaft --- und der Missbrauch von Benutzerdaten ist eine l�stige Nebens�chlichkeit, die man leider nicht abstellen kann.

Es ist nicht offensichtlich, aber es scheint hier durch, dass die Plattform Facebook nicht als Produkt einer Firma angesehen wird, das von Vielen auf freiwilliger Basis genutzt wird, sondern als Institution, als Autorit�t, die da ist und mit der man sich abfinden muss.

Dieser Mangel an Distanz und Hinterfragung ist von Seiten der Medien unverantwortlich, insbesondere, da es durchaus differenzierte und kritische Meinungen zum Thema Facebook, vor Allem im Hinblick auf Datenschutz und Privatsph�re, gibt. 
Das eigentliche Paradoxon ist jedoch nicht die Akzeptanz von Facebok, Twitter o.�. als Nachrichtenquelle, sondern die Tatsache, dass die Vertraunsw�rdigkeit einer solchen Quelle als gleichwertig gegen�ber beispielsweise einer Zeitung angesehen wird. 

�hnliche Ausma�e nahmen die Diskussionen um die Plagiatsvorw�rfe gegen den ehemaligen Verteidigungsminister Karl-Theodor zu Guttenberg an. W�hrend die Erfassung und Bearbeitung verd�chtiger Stellen der Doktorarbeit in einem �ffentlich zug�nglichen und von jedermann bearbeitbaren Wiki stattfanden (GuttenPlag\autocite{spon_guttenplag}), bildete sich ein gro�er Facebook-Zusammenschluss, der das Ende der ,Hetzjagd' auf den Politiker forderte. Von den traditionellen Medien wurden diese Proteste teilweise als eine wirklich stattfindende Demonstration angesehen und verfolgt. 
So wurden einzelne Beitr�ge, die Mitglieder der im Rahmen der Pro-Guttenberg-Facebookgruppe verfassten, wie \emph{vox populi}-Beitr�ge \marginpar{\emph{vox populi} -- lat. ,Stimme des Volkes'} bezeichnet behandelt, Aussagen von Demonstranten oder Passanten, die �blicherweise von Journalisten bei Demonstrationen oder anderen �ffentlichen Veranstaltungen erfragt werden. Die Ern�chterung folgte dann auf dem Fu�e: ein Aufruf der Pro-Guttenberg-Facebookgemeinde, zu einer Protestkundgebung auf die Stra�e zu gehen, sto� auf �u�erst geringe Resonanz.\footcite{taz_guttendemo} 

Durch F�lle wie diesen wird deutlich, wie sehr die Immaterialit�t des Internets die Hemmschwelle f�r Kommunikation senkt. Ein Klick auf den \enquote{Gef�llt mir}--Knopf ist schnell getan, doch das Besuchen einer Demonstration erfordert weit mehr Engagement, als der durchschnittliche Internetbenutzer aufbringen kann und will.

i.	Netzwerke nicht als Produkt/Selbstzweck, sondern als Plattform
ii.	Bsp: Stars/Firmenvertreter/
c.	�Digital Natives' / �Digital Immigrants'

 weil wie auch bei einschl�gigen Online-Enzyklop�dien jeder einen derartigen Beitrag verfassen kann.


\chapter{Neu erschlossene Bereiche}
a.	Einfach zug�nglich -> Diskussionsplattform
b.	Politische Einflussnahme
i.	Libyen/�gypten
ii.	Atomkraft?
iii.	Open Government


\chapter{Probleme}

\section{Zentralisierte Organisation}
-i.	Diaspora
\section{Privatsph�re}

\chapter{Ausblick}



%%%%%%%%%%%%%%%%%%%%%%%%%%%%%%%%%%%%%%%%%%%%%%%%%%%%%%%%%%%%%
%% LITERATURVERZEICHNIS
%%%%%%%%%%%%%%%%%%%%%%%%%%%%%%%%%%%%%%%%%%%%%%%%%%%%%%%%%%%%%
\clearpage
\addtocontents{toc}{\protect\vspace*{\baselineskip}} % Ein kleiner Abstand zu den Kapiteln im Inhaltsverzeichnis (toc)
\addcontentsline{toc}{chapter}{Literaturverzeichnis}
\nocite{*} %Auch nicht-zitierte BibTeX-Eintr�ge werden angezeigt.

\printbibliography 

%% Abbildungsverzeichnis
%\clearpage
%\addcontentsline{toc}{chapter}{Abbildungsverzeichnis}
%\listoffigures

%% Tabellenverzeichnis
%\clearpage
%\addcontentsline{toc}{chapter}{Tabellenverzeichnis}
%\listoftables


%%%%%%%%%%%%%%%%%%%%%%%%%%%%%%%%%%%%%%%%%%%%%%%%%%%%%%%%%%%%%
%% ANH�NGE
%%%%%%%%%%%%%%%%%%%%%%%%%%%%%%%%%%%%%%%%%%%%%%%%%%%%%%%%%%%%%
\appendix

\end{document}

